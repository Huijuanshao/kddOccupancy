\documentclass[sigconf]{acmart}

\usepackage{booktabs} % For formal tables
\usepackage{times}
\usepackage{helvet}
\usepackage{courier}
\usepackage{graphicx}
\usepackage{multirow}
\usepackage{algorithm}
\usepackage{algorithmic}
\usepackage{mathrsfs}
\usepackage{comment}
%\newtheorem{definition}{Definition}
%\usepackage{algorithmicx}
%\usepackage{algorithm} % http://ctan.org/pkg/algorithms

% Copyright
%\setcopyright{none}
%\setcopyright{acmcopyright}
%\setcopyright{acmlicensed}
\setcopyright{rightsretained}
%\setcopyright{usgov}
%\setcopyright{usgovmixed}
%\setcopyright{cagov}
%\setcopyright{cagovmixed}


% DOI
%\acmDOI{10.475/123_4}

% ISBN
%\acmISBN{123-4567-24-567/08/06}

%Conference
%\acmConference[WOODSTOCK'97]{ACM Woodstock conference}{July 1997}{El
%  Paso, Texas USA} 
%\acmYear{2017}
\copyrightyear{2017}

%\acmPrice{15.00}


\begin{document}
\title{Temporal Motif Mining Approaches for Smart Homes}
%\titlenote{Produces the permission block, and copyright information}
%\subtitle{Extended Abstract}
%\subtitlenote{The full version of the author's guide is available as
%  \texttt{acmart.pdf} document}

\author{
  Huijuan Shao\textsuperscript{1,2}, Yaowei Li\textsuperscript{3}, Fei Li\textsuperscript{2}, Erin Griffiths \textsuperscript{4}, Kamin Whitehouse\textsuperscript{4}, Naren Ramakrishnan\textsuperscript{1,2} \\
  \textsuperscript{1}Discovery Analytics Center, Virginia Tech, Blacksburg, VA 24061 \\ %\vspace{0.1cm}
  \textsuperscript{2}Department of Computer Science, Virginia Tech, Blacksburg, VA 24061 \\ %\vspace{0.1cm}
    \textsuperscript{3} Game source, McLean, VA 22101 \\
  \textsuperscript{4}Department of Computer Science, University of Virginia, VA 22904 \\ %\vspace{0.1cm}
}




% The default list of authors is too long for headers}
\renewcommand{\shortauthors}{B. Trovato et al.}


\begin{abstract}
With the advent of modern sensor technologies, 
significant opportunities have opened up to help conserve energy in 
residential and commercial buildings. Moreover, the rapid \emph{urbanization} we are witnessing requires optimized energy distribution. 
This paper focuses on two sub-problems in improving energy conservation; \emph{energy disaggregation and occupancy prediction}. 
Energy disaggregation attempts to 
separate the energy usage 
of each circuit or each electric device in a building 
using only aggregate electricity usage information from 
the meter for the whole house. 
The second problem of occupancy prediction can be accomplished using non-invasive indoor activity tracking to 
predict the locations of people inside a building. 
We cast both problems as \emph{temporal mining problems}. We exploit motif mining with constraints to distinguish devices with multiple states, which helps tackle the energy disaggregation problem. Our
results reveal that motif mining is adept at distinguishing
devices with multiple power levels and at disentangling the
combinatorial operation of devices.
For the second problem we propose time-gap constrained episode mining to detect 
activity patterns followed by the use of a mixture of episode generating HMM (EGH) models 
to predict home occupancy.  
Finally, we demonstrate that the mixture EGH
model can also help predict the location of a person to 
address non-invasive indoor activities tracking. 
\end{abstract}

%
% The code below should be generated by the tool at
% http://dl.acm.org/ccs.cfm
% Please copy and paste the code instead of the example below. 
%


\keywords{ACM proceedings, \LaTeX, text tagging}


\maketitle

\section{Introduction}

Modeling the activity of daily life (ADL) has become a fertile research topic, 
satisfying the demand for a comfortable home life
at a lower cost. 
Since heating and cooling our living spaces consumes $\sim$53\% of the total electrical used
by an average household, 
automating the operation of HVAC devices to minimize energy usage is clearly important. 
One of the crucial components required to achieve this goal is the ability
to accurately model and predict the occupancy of a home. 
Supervised learning approaches based on the analysis of indoor temperature~\cite{kleiminger2014predicting}, 
GPS data from smart phones~\cite{koehler2013therml}, 
historical electricity consumption data~\cite{erickson2010occupancy} and 
sensor data generated by tracking indoor activities~\cite{scott2011preheat,alrazgan2011learning} 
have all been shown to be effective ways to approach this prediction problem. 
Prediction of occupancy using sensor data has been broadly studied 
generally by capturing daily activities like room occupancy within the house, 
usage of electrical devices, 
and usage of water systems to model occupancy~\cite{mahmoud2013behavioural,erickson2010occupancy,beltran2014optimal} 
and using these results to automate the control of the HVAC system. 

Although supervised learning techniques---kNN~\cite{scott2011preheat}, 
neural network~\cite{mahmoud2013behavioural} and Markov model~\cite{erickson2010occupancy}---have all shown to be effective, 
the fine-grained details of household activities represented as a time series have not yet been fully utilized. 
Daily activities such as waking up, cooking, washing, and commuting to and from work/school
have different patterns based on the day of week. 
For instance, the schedule on a working day is significantly different from that on a weekend 
or a holiday. 
Thus, this scenario leads itself to episode mining analysis such as the approach adopted in this study
to predict household occupancy. 
Applying this strategy of episode mining for occupancy prediction has three advantages. 
First, episode mining, a temporal mining approach, mines the data according to the time 
distribution for each type of activity. 
Second, it can be used to build an activity scenario and connect different episodes with 
a probabilistic hidden Markov model (HMM). 
Unlike earlier models, 
the time and order of each kind of activity can be fully utilized using this approach.
Third, the algorithm predictions are based on a scenario-based probabilistic model episode 
generative HMM (EGH). The prediction accuracy is consequently a marked improvement over that achieved by existing models. 

The contributions of this study can be summarized in terms of the way it addresses the three questions below. 
\begin{enumerate}
\item How can we mine for meaningful scenarios? 
Episode mining can mine many frequent episodes, but not all these episodes will be useful 
for occupancy prediction. 
By narrowing down the episodes included in the analysis according to their start state, 
end time, event dwelling time and the gap between two activities, 
we can interpret these episodes more accurately thus pinpoint the episodes more likely to convey useful information . 
\item How can we predict occupancy more accurately?
Our dataset comprises detailed information regarding the various activities of a household 
tracked as a time series on a daily basis. 
Thus our episodes contain richly detailed information based on the occupancy/unoccupancy status of the household. 
Since we are mining episodes from this data, 
the accuracy of occupancy prediction improves significantly. 
\item  Can the new approach proposed here help reduce electricity usage in the home?
The prediction gives at least 15 minutes advance notice of the time a person 
leaves or returns home. 
By connecting this prediction result to an automatic HVAC control system, the HVAC can be turned on or off  ahead of the occupancy change. 
Since this means that the HVAC does not heat or cool the home when there is no one home, this can substantially reduce the household's electricity usage. 
\end{enumerate}

\iffalse
Next, 
we first discuss the time-gap constraint episode mining 
model and the mixture model, 
and how to predict the target event in section 4. 
Then, 
in section 5 we will show 
the experimental results. 
\fi

%Other similar research field is the mobility of person, from work to home. 
%\cite{baumann2013influence} is similar to our work but from work  to home. It utilizes the second-order Markov Model. MAJOR is the novel approach. 

%\cite{baumann2013long} gives a good definition on the prediction. It is very related to our work. 

%\cite{kleiminger2013inferring} has a home set algorithm related to time/day. It has a beautiful figure on the household occupancy. 







\include{sec/prior}
%\include{sec/multivariateApp}
\include{sec/predictionApp}
%\include{sec/disaggResults}
\include{sec/predictionResults}
\section{Conclusion}
Residential occupancy prediction is a hot research topic supporting efforts to control HVAC systems more efficiently, 
with a consequent reduction in energy consumption. 
Increasing the accuracy of occupancy prediction allows these saving to be made 
without sacrificing the comfort of those living inside 
the home. 
In order to achieve the best prediction results, 
we propose integrating the mixture EGH model and 
kNN to create a new hybrid approach.

Our work differs from previous research based on the main contributions listed below:
\begin{enumerate}
\item We formulate the problem as one of temporal mining, 
where the activities inside the building are abstracted as episodes, 
and each episode is connected via an episode generative HMM model.
\item We mine the activity patterns according to the times and the gaps between them: 
both the duration of each type of 
activity, and the gap between two consecutive events are limited within an appropriate range. 
This range is extracted from historical data according to the day of the week and holidays.
\item Our hydrid prediction solution performs best for workday occupancy prediction: 
in the case of normal activities, a mixture EGH model is applied and
in the case of abnormal events, kNN is utilized,  
as this is generally considered a benchmark in occupancy prediction problems. 
\end{enumerate}
%In this paper, we propose the mixture EGH model and compare it with two other 
%benchmark models, probability density function and kNN approach. 
%The results show that it generally performs better than kNN to predict the 
%occupancy and un-occupancy states in the workdays. 
%The mixture model predicts well for the period of after person getting up and before person 
%going out. 
%The coefficient of the episode generative HMM models helps 
%predict the exact leaving time. 
%However in the case of abnormal events, 
%kNN performs good because it can average the 
%historical data. 
%Even if there is an abnormal day, 
%kNN can leverage it. 

In future work, we plan to continue working on improving holiday occupancy prediction. 
This is a  more intractable problem because 
the occupancy patterns for these days are completely different. 
For example, on certain days, a person may never leave the house. 
Therefore the occupancy prediction for holidays probably depends more on the date than on the indoor activities performed. 
We also plan to apply this new temporal mining approach to GPS datasets~\cite{koehler2013therml}
to test the effectiveness of occupancy prediction with different kinds of data. 

\bibliographystyle{ACM-Reference-Format}
\bibliography{references} 

\end{document}
