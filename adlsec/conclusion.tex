\section{Conclusion}
Residential occupancy prediction is a hot research topic supporting efforts to control HVAC systems more efficiently, 
with a consequent reduction in energy consumption. 
Increasing the accuracy of occupancy prediction allows these saving to be made 
without sacrificing the comfort of those living inside 
the home. 
In order to achieve the best prediction results, 
we propose integrating the mixture EGH model and 
kNN to create a new hybrid approach.

Our work differs from previous research based on the main contributions listed below:
\begin{enumerate}
\item We formulate the problem as one of temporal mining, 
where the activities inside the building are abstracted as episodes, 
and each episode is connected via an episode generative HMM model.
\item We mine the activity patterns according to the times and the gaps between them: 
both the duration of each type of 
activity, and the gap between two consecutive events are limited within an appropriate range. 
This range is extracted from historical data according to the day of the week and holidays.
\item Our hydrid prediction solution performs best for workday occupancy prediction: 
in the case of normal activities, a mixture EGH model is applied and
in the case of abnormal events, kNN is utilized,  
as this is generally considered a benchmark in occupancy prediction problems. 
\end{enumerate}
%In this paper, we propose the mixture EGH model and compare it with two other 
%benchmark models, probability density function and kNN approach. 
%The results show that it generally performs better than kNN to predict the 
%occupancy and un-occupancy states in the workdays. 
%The mixture model predicts well for the period of after person getting up and before person 
%going out. 
%The coefficient of the episode generative HMM models helps 
%predict the exact leaving time. 
%However in the case of abnormal events, 
%kNN performs good because it can average the 
%historical data. 
%Even if there is an abnormal day, 
%kNN can leverage it. 

In future work, we plan to continue working on improving holiday occupancy prediction. 
This is a  more intractable problem because 
the occupancy patterns for these days are completely different. 
For example, on certain days, a person may never leave the house. 
Therefore the occupancy prediction for holidays probably depends more on the date than on the indoor activities performed. 
We also plan to apply this new temporal mining approach to GPS datasets~\cite{koehler2013therml}
to test the effectiveness of occupancy prediction with different kinds of data. 