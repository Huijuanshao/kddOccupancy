\section{Introduction}

Modeling the activity of daily life (ADL) has become a fertile research topic, 
satisfying the demand for a comfortable home life
at a lower cost. 
Since heating and cooling our living spaces consumes $\sim$53\% of the total electrical used
by an average household, 
automating the operation of HVAC devices to minimize energy usage is clearly important. 
One of the crucial components required to achieve this goal is the ability
to accurately model and predict the occupancy of a home. 
Supervised learning approaches based on the analysis of indoor temperature~\cite{kleiminger2014predicting}, 
GPS data from smart phones~\cite{koehler2013therml}, 
historical electricity consumption data~\cite{erickson2010occupancy} and 
sensor data generated by tracking indoor activities~\cite{scott2011preheat,alrazgan2011learning} 
have all been shown to be effective ways to approach this prediction problem. 
Prediction of occupancy using sensor data has been broadly studied 
generally by capturing daily activities like room occupancy within the house, 
usage of electrical devices, 
and usage of water systems to model occupancy~\cite{mahmoud2013behavioural,erickson2010occupancy,beltran2014optimal} 
and using these results to automate the control of the HVAC system. 

Although supervised learning techniques---kNN~\cite{scott2011preheat}, 
neural network~\cite{mahmoud2013behavioural} and Markov model~\cite{erickson2010occupancy}---have all shown to be effective, 
the fine-grained details of household activities represented as a time series have not yet been fully utilized. 
Daily activities such as waking up, cooking, washing, and commuting to and from work/school
have different patterns based on the day of week. 
For instance, the schedule on a working day is significantly different from that on a weekend 
or a holiday. 
Thus, this scenario leads itself to episode mining analysis such as the approach adopted in this study
to predict household occupancy. 
Applying this strategy of episode mining for occupancy prediction has three advantages. 
First, episode mining, a temporal mining approach, mines the data according to the time 
distribution for each type of activity. 
Second, it can be used to build an activity scenario and connect different episodes with 
a probabilistic hidden Markov model (HMM). 
Unlike earlier models, 
the time and order of each kind of activity can be fully utilized using this approach.
Third, the algorithm predictions are based on a scenario-based probabilistic model episode 
generative HMM (EGH). The prediction accuracy is consequently a marked improvement over that achieved by existing models. 

The contributions of this study can be summarized in terms of the way it addresses the three questions below. 
\begin{enumerate}
\item How can we mine for meaningful scenarios? 
Episode mining can mine many frequent episodes, but not all these episodes will be useful 
for occupancy prediction. 
By narrowing down the episodes included in the analysis according to their start state, 
end time, event dwelling time and the gap between two activities, 
we can interpret these episodes more accurately thus pinpoint the episodes more likely to convey useful information . 
\item How can we predict occupancy more accurately?
Our dataset comprises detailed information regarding the various activities of a household 
tracked as a time series on a daily basis. 
Thus our episodes contain richly detailed information based on the occupancy/unoccupancy status of the household. 
Since we are mining episodes from this data, 
the accuracy of occupancy prediction improves significantly. 
\item  Can the new approach proposed here help reduce electricity usage in the home?
The prediction gives at least 15 minutes advance notice of the time a person 
leaves or returns home. 
By connecting this prediction result to an automatic HVAC control system, the HVAC can be turned on or off  ahead of the occupancy change. 
Since this means that the HVAC does not heat or cool the home when there is no one home, this can substantially reduce the household's electricity usage. 
\end{enumerate}

\iffalse
Next, 
we first discuss the time-gap constraint episode mining 
model and the mixture model, 
and how to predict the target event in section 4. 
Then, 
in section 5 we will show 
the experimental results. 
\fi

%Other similar research field is the mobility of person, from work to home. 
%\cite{baumann2013influence} is similar to our work but from work  to home. It utilizes the second-order Markov Model. MAJOR is the novel approach. 

%\cite{baumann2013long} gives a good definition on the prediction. It is very related to our work. 

%\cite{kleiminger2013inferring} has a home set algorithm related to time/day. It has a beautiful figure on the household occupancy. 






